\chapter{Maps}
The main reason for introducing smooth structures was to enable us to define
smooth functions on manifolds and smooth maps between manifolds. We generally
reserve the term "function" for a map whose range is $\R$ or $\R^n$. The word
"map" or "mapping" can mean any type of map betwen arbitrary manifolds. 

We then study diffeomorphisms, which are bijective smooth maps with smooth inverses.
The main objects of study in smooth manifold theory are properties that are 
invariant under diffeomorphisms. Later in the chapter, we study covering maps, and
their relationship to the continuous covering maps in topology; and we introduce
Lie groups, which are smooth manifold that are also groups in which multiplication
and inversion are smooth maps. At the end, we introduce some powerful tools for
smoothly piecing together local smooth objects, called bump functions and partitions
of unity. They will be used for building global smooth objects out of ones that 
are initially defined only locally.

\section{Smooth functions and maps}
\begin{definition}
If $M$ is a smooth manifold, a function $f:M\rightarrow\Re^k$ is said to be smooth if,
for every smooth char $(U,\varphi)$ on $M$, the composite function $f\circ\varphi^{-1}$
is smooth on the open subset $\varphi(U)\subset \R^n$. 
\end{definition}
Although by definition smoothness of $f$ means that its composition with \emph{every}
smoooth coordinate map is smooth, in pratice it suffices to check smoothness in each
of the charts of some smooth atlas, as the next lemma shows.
\begin{lem}
Suppose $\{(U_\alpha,\varphi_\alpha)\}$ is a smooth atlas for $M$. If $f:M\rightarrow\R^k$
is a function s.t. $f\circ\varphi_{\alpha}^{-1}$ is smooth for each $\alpha$, then $f$ is
smooth.
\end{lem}
\begin{proof}
We just need to check that $f\circ\varphi^{-1}$ is smooth for any smooth chart $(U,\varphi)$
on $M$. It suffices to show it is smooth in a neighborhood of each point $x=\varphi(p)\in 
\varphi(U)$. For any $p\in U$, there is a chart $(U_\alpha,\varphi_\alpha)$ in the atlas whose
domain contains $p$. Since $(U,\varphi)$ is smoothly compatible with $(U_\alpha,\varphi_\alpha)$,
the transition map $\varphi_\alpha\circ\varphi^{-1}$ is smooth on its domain of definition, 
which includes $x$. Thus $f\circ\varphi^{-1}=(f\circ\varphi_{\alpha}^{-1})\circ(\varphi_\alpha
\circ\varphi^{-1})$ is smooth in a neighborhood of $x$.
\end{proof}
Given a function $f$ and a chart $(U,\varphi)$, the function $\hat{f}:\varphi(U)\rightarrow\R^k$
defined by $\hat{f}(x)=f\circ\varphi^{-1}(x)$ is called the \emph{coordinate representation} of $f$.
\begin{definition}
Let $M,N$ be smooth manifolds, a map $F:M\rightarrow N$ is smooth if, for any smooth charts
$(U,\varphi)$ for $M$ and $(V,\phi)$ for $N$, the composite map $\phi\circ F\circ\varphi^{-1}$
is smooth from $\varphi(U\cap F^{-1}(V))$ to $\phi(V)$.
\end{definition}
\begin{exer}
Let $F:M\rightarrow N$ be a map between smooth manifolds, and suppose each point $p\in M$ has a 
neighborhood $U$ s.t. $F\mid_U$ is smooth. Show that $F$ is smooth.
\end{exer}
Similarly, we call $\hat{F}=\phi\circ F\circ\varphi^{-1}$ the coordinate representation of $F$.
\begin{lem}\label{lem:22}
Let $F:M\rightarrow N$ be the map of two smooth manifolds. If $\{(U_\alpha,\varphi_\alpha)\}$ and
$\{(V_\beta,\phi_\beta)\}$ are smooth atlases, and if for each $\alpha,\beta$, $\phi_\beta\circ
F\circ\varphi_{\alpha}^{-1}$ is smooth on its domain of definition, then $F$ is smooth.
\end{lem}
\begin{exer}
Prove Lemma \ref{lem:22}.
\end{exer}
\begin{lem}
Any composition of smooth maps between manifolds is smooth.
\end{lem}
\begin{proof}
Given smooth maps $F:M\rightarrow N$ and $G:N\rightarrow P$, let $(U,\varphi)$ and $(V,\phi)$ be
any charts for $M,N$. We need to show that $\phi\circ(G\circ F)\circ\varphi^{-1}$ is smooth where
it is defined, namely on $\varphi(U\cap (G\circ F)^{-1}(V))$. For any point $p\in U\cap (G\circ F)
^{-1}(V)$, there is a chart $(W,\theta)$ for $N$ s.t. $F(p)\in W$. Smoothness of $F$ and $G$ means
that $\theta\circ F\circ\varphi^{-1}$ and $\phi\circ G\circ\theta^{-1}$ are smooth where they are
defined, and therefore $\phi\circ(G\circ F)\circ\varphi^{-1}=(\phi\circ G\circ\theta^{-1})\circ(
\theta\circ F\circ\varphi^{-1})$ is smooth.
\end{proof}
\begin{exer}
Show that a map $F:N\rightarrow M_1\times\dots \times M_k$ is smooth iff each of the "component maps"
$F_i=\pi_i\circ F:N\rightarrow M_i$ is smooth. (Here $\pi_i:M_1\times\dots \times M_k\rightarrow M_i$
is the projection onto the i-th factor.)
\end{exer}
\begin{definition}
A diffeomorphism between two manifolds is a smooth map $F:M\rightarrow N$ that has a smooth inverse.
This is symbolized by $M\approx N$. 
\end{definition}
More generally, $F$ is called a local diffeomorphism if every point $p\in U$ s.t. $F(U)$ is an open
set in $N$ and $F:U\rightarrow F(U)$ is a diffeomorphism.
\begin{exer}
Show that a map is a diffeomorphism iff it is a bijective local diffeomorphism.
\end{exer}
Two smooth manifolds are essentially indistinguishable if they are diffeomorphic. One question that
naturally arises is to what extent a smooth structure on a given topological manifold might be unique.
There are really two different question here: The first is whether a given manifold M admits distinct
smooth structures, and the second is whether it admits smooth structures that are not diffeomorphic to
each other. In general, a given topological manifold will admit very many distinct smooth structure.
For the second question, it  follows from the work of Edwin Moise and James Munkres that every topological
manifold of dimension k ($k\leqslant 3$) has a smooth structure that is unique up to diffeomorphism.
The analogous question in higher dimensions turns out to be quite deep and is still largely unanswered.
The problem of identifying the number of smooth structures (if any) on topological 4-manifolds is an
active subject of current research.
\section{Smooth covering maps}
\section{Lie Group}
\section{Bump Function and Partition of Unity}
\section{Problems}