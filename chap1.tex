\chapter{Manifolds}
A smooth manifold can be seen as a set with two layers of structure: first a toplogy,
ten a smooth structure.

\section{Topological Manifolds}
\begin{definition}[\emph{Topological Manifold}] 
Suppose $M$ is a topological space, M is a \emph{topological n-manifold} if
\begin{enumerate}
\item[$\bullet$] $M$ is a \emph{Hausdorff space}: For every pair of points $p,q \in M$, 
there are disjoint open subsets $U,V\subset M$such that $p\in U$ and $q\in V$.
\item[$\bullet$] $M$ is \emph{second countable}: There exists a coutable basis for 
the topologyof $M$.
\item[$\bullet$] $M$ is \emph{locally Euclidean of dimension n}: Every point has a 
neighborhood that is homeomorphic to an open subset of $\R^n$. 
\end{enumerate}
\end{definition}
Requreing that manifolds share these properties helps to ensure that manifolds behave
in the ways we expect from our experience with Euclidean space. For example, it is easy
to verify that in a Hausdorff space, onepoint sets are closed and limits of convergent
sequences are unique. The motivation for second countability is a bit less evident, 
but it will have important consequences throughout the book, beginning with the existence
of partitions of unity.
\begin{lem}
Let M be a second countable topological space. Then every open cover of M has a countable
subcover. Proof can be found in [Lee00, Lemma2.15].
\end{lem}
\begin{definition}[\emph{coordinate chart}]
A coordinate chart on M is a pair $(U,\varphi)$, where $U$ is an open subset of M and 
$\varphi:U\rightarrow \widetilde{U}\subset \R^n$. $U$ is a coordinate domain, $\varphi$
is coordinate map.
\end{definition}
\section{Smooth Structure}
\section{Examples}
\section{Local Coordinate Representation}
\section{Manifolds with Boundary}
\section{Problems}