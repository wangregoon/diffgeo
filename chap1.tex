\chapter{Manifolds}
A smooth manifold can be seen as a set with two layers of structure: first a toplogy,
ten a smooth structure.

\section{Topological Manifolds}
\begin{definition}[\emph{Topological Manifold}] 
Suppose $M$ is a topological space, M is a \emph{topological n-manifold} if
\begin{enumerate}
\item[$\bullet$] $M$ is a \emph{Hausdorff space}: For every pair of points $p,q \in M$, 
there are disjoint open subsets $U,V\subset M$such that $p\in U$ and $q\in V$.
\item[$\bullet$] $M$ is \emph{second countable}: There exists a coutable basis for 
the topologyof $M$.
\item[$\bullet$] $M$ is \emph{locally Euclidean of dimension n}: Every point has a 
neighborhood that is homeomorphic to an open subset of $\R^n$. 
\end{enumerate}
\end{definition}
Requreing that manifolds share these properties helps to ensure that manifolds behave
in the ways we expect from our experience with Euclidean space. For example, it is easy
to verify that in a Hausdorff space, onepoint sets are closed and limits of convergent
sequences are unique. The motivation for second countability is a bit less evident, 
but it will have important consequences throughout the book, beginning with the existence
of partitions of unity.
\begin{lem}
Let M be a second countable topological space. Then every open cover of M has a countable
subcover. Proof can be found in [Lee00, Lemma2.15].
\end{lem}
\begin{exer}
Show that equivalent definitions of locally Euclidean spaces are obtained if, instead of
requiring $U$ to be homeomorphic to an open subset of $\Re^n$, we require it to be homeomorphic
to an open ball in $\Re^n$, or to $\Re^n$ itself.
\end{exer}
\begin{exer}
Show that any topological subspace of a Hausdorff space is Hausdorff, and any finite product of
Hausdorff space is Hausdorff.
\end{exer}
\begin{exer}
Show that any topological subspace of a second countable space is second countable, and any finite
product of second countable spaces is second countable.
\end{exer}
\begin{definition}[\emph{coordinate chart}]
A coordinate chart on M is a pair $(U,\varphi)$, where $U$ is an open subset of M and 
$\varphi:U\rightarrow \widetilde{U}\subset \R^n$. $U$ is a coordinate domain, $\varphi$
is coordinate map.
\end{definition}
\begin{example}
Let $\mathcal{S}^n$ denote the n-sphere, which the set of unit-length vector in $\R^{n+1}$
$$ \mathcal{S}^n = \{x\in \R^{n+1}: |x|=1\}$$
It is Hausdorff and second countable because it is a subspace of $\R^n$. To Show that it
is locally Euclidean, let 
$$ U_i^{\pm}=\{(x^1,\dots,x^{n+1})\in \mathcal{S}^n : x_{+}^i > 0, x_{-}^i < 0 \}$$
Define maps $\varphi_i^{\pm}:U_i^{\pm}\rightarrow\R^n$ by
$$ \varphi_i^{\pm}(x^1,\dots,x^{n+1})=(x^1,\dots,x^{i-1},\pm\sqrt{1-|x|^2},x^{i+1},\dots,x^{n+1})$$
Since every point of $\mathcal{S}^n$ is in the domain of one of these $2n+2$ charts, so
it is locally a topological n-manifold.
\end{example}
\begin{example}
The n-dimensional real projective space $\mathbb{P}^n$ is defined as the set of linear 
subspace of $\R^{n+1}$. We give it the quotient topology determined by the natural map
$\pi:\R^{n+1}\backslash\{0\}\rightarrow\mathbb{P}^n$ sending each point $x$ to the line through
$x$ and 0. For any point $x$, let $[x]=\pi(x)$ denote the equivalence class of x in 
$\mathbb{P}^n$. For each i, let $\tilde{U}_i\subset\R^{n+1}\backslash\{0\}$ be the set 
where $x^i\neq 0$, andlet $U_i=\pi(\tilde{U}_i)\subset \mathbb{P}^n$. Define a map 
$\varphi:U_i\rightarrow\R^n$by $$ \varphi_i[x^1,\dots,x^{n+1}]=\left(\frac{x^1}{x^i},
\dots,\frac{x^{i-1}}{x^i},\frac{x^{i+1}}{x^i},\dots,\frac{x^{n+1}}{x^i},\right)$$
This map is well-defined because its value is unchanged by multiplying $x$ by a nonzero
constant, and it is continuous because $\varphi\circ\pi$ is continuous.
\end{example}

\section{Smooth Structure}
However, in the entire theory of topological manifolds, there is no mention of calculus.
There is a good reason for this: Whatever sense we might try to make derivatives of
functions or curves on a manifold, the cannot be invariant under homoemorphisms.
To make sense of derivatives of functions, curves, or maps, we will need to introduce a
new kind of manifold called a "smooth manifold"("smooth" means $C^\infty$ and property
of "smoothness" is invariant under homeomorphisms).

For example, a function $f:M\rightarrow\R$ is smooth iff the composite function 
$f\circ\varphi^{-1}:\tilde{U}\rightarrow\R$ is smooth. But this will make sense only if
this property is independent of the choice of coordinate chart. To guarantee this, we
will restrict our attention to "smooth charts".

If $(U,\varphi),(V,\phi)$ are two charts such that $U\cap V\neq \emptyset$, then the 
composite map $\phi\circ\varphi^{-1}:\varphi(U\cap V)\rightarrow\phi(U\cap V)$ is a 
composition of homeomorphisms. Two charts are said to be \emph{smoothly compatible} if
either $U\cap V = \emptyset$ or the transition map $\phi\circ\varphi^{-1}$ is a diffeomorphism.

We define an \emph{atlas} for M to be a collection of charts whose domains cover M.
An atlas $\mathcal{A}$ is called a \emph{smooth atlas} if any two charts in $\mathcal{A}$
are smoothly compatible with each other.

Our plan is to define a "smooth structure" on M by giving a smooth atlas, and to define a 
function to be smooth in the atlas. There is one minor technical problem with this approach:
In general, there will be many possible choices of atlas that give the "smae" smooth 
structure, in that they all determine the same collection of smooth functions on M.
\emph{maximal atlas} would be a solution to this. A smooth atlas $\mathcal{A}$ on M is 
maximal if it is not contained in any strictly larger smooth atlas.

Now, we can define that a \emph{smooth structure} on a topological n-manifold M is a maximal
smooth atlas. A \emph{smooth manifold} is a pair $(M,\mathcal{A})$. We emphasize that a
smooth structure is an additional piece of data that must be added to a topological manifold
before we are entitled to talk about a "smooth manifold." In fact, a given topological manifold
may have many different smooth sturctures or it contains no smooth structures at all.

It is generally not very convenient to define a smooth structure by explicitly describing a
maximal smooth atlas, because such an atlas contains very many charts. Fortunately, we need
only specify some smooth atlas, as the next lemma shows.
\begin{lem}\label{lem:smooth_atlas}
Let M be a topological manifold
\begin{enumerate}
\item[(a)] Every smooth atlas for M is contained in a unique maximal smooth atlas.
\item[(b)] Two smooth atlases for M determine the same maximal smooth atlas iff their union is
a smooth atlas.
\end{enumerate}
\end{lem}
\begin{proof}
Let $\mathcal{A}$ be a smooth atlas for M, and let $\overline{\mathcal{A}}$ denote the 
set of all charts that are smoothly compatible with every chart in $\mathcal{A}$. To 
show that $\overline{\mathcal{A}}$ is a smooth atlas, we need to show that any two
charts of $\overline{\mathcal{A}}$ are compatible with each other, which is to say that
for any $(U,\varphi),(V,\phi)\in \overline{\mathcal{A}}, \phi\circ\varphi^{-1}$ is
smooth. Let $x=\varphi(p)\in \varphi(U\cap V)$ be arbitrary. Because the domains of the
charts in $\mathcal{A}$ cover M, the is some chart $(W,\theta)\in \mathcal{A}$ s.t. 
$p\in W$. Since every chart in $\overline{\mathcal{A}}$ is smoothly compatible with 
$(W,\theta)$, both the map $\theta\circ\varphi^{-1}$ and $\phi\circ\theta^{-1}$ are 
smooth where they defined. Since $p\in U\cap V \cap W$, it follows that $\phi\circ\varphi^{-1}
=(\phi\circ\theta^{-1})\circ\theta\circ\varphi^{-1}$ is smooth on a neighborhood of $x$.
Thus $\phi\circ\varphi^{-1}$ is smooth in a neighborhood of each point in $\varphi(U\cap V)$.
Therefore $\overline{\mathcal{A}}$ is a smooth atlas. To check that it is maximal, just
note that any chart that is smoothly compatible with every chart in $\overline{\mathcal{A}}$
must in particular be smoothly compatible with every chart in $\mathcal{A}$, so it is already
int $\overline{\mathcal{A}}$. This proves the existence of a maixmal smooth atlas containing
$\mathcal{A}$. If $\mathcal{B}$ is any other maximal smooth atlas containing $\mathcal{A}$,
each of its charts is smoothly compatible with each chart in $\mathcal{A}$, so $\mathcal{B}\subset
\overline{\mathcal{A}}$. By maximality, $\mathcal{B}=\overline{\mathcal{A}}$.
\end{proof}
\begin{exer}
Prove Lemma \ref{lem:smooth_atlas}(b).
\end{exer}
%\section{Examples}
%\section{Local Coordinate Representation}
%\section{Manifolds with Boundary}
\section{Problems}
\begin{problem}
Let X be the set of all points $(x,y)\in \R^2$ s.t. $y=\pm 1$, and let M be the 
quotient of X by the equivalence relation generated by $(x,-1)\sim (x,1)$ for all
$x\neq 0$. Show that M is locally Euclidean and second countable, but not Hausdorff.
[This space is called the \textbf{line with  two origins}]
\end{problem}
\begin{problem}
Show that the disjoint union of uncountably many copies of $\R$ is locally Euclidean
and Hausdorff, but not second countable.
\end{problem}
\begin{problem}
Let $N=(0,\dots,0,1)$ be the "north pole" and $S=-N$ the "south pole." Define steregraphic
projection $\sigma:\mathbb{S}^n\backslash \{N\}\rightarrow\R^n$ by $$ \sigma(x^1,\dots,x^{n+1})
=\frac{(x^1,\dots,x^n)}{1-x^{n+1}}$$ Let $\tilde{\sigma}(x)=\sigma(-x)$ for $x\in \mathbb{S}^n
\backslash \{S\}$.
\begin{enumerate}
\item[(a)] Show that $\sigma$ is bijective, and $$\sigma^{-1}(u^1,\dots,u^n)=\frac{(2u^1,\dots,2u^n,
|u|^2-1)}{|u|^2+1}$$
\item[(b)] Compute the transition map $\tilde{\sigma}\circ\sigma^{-1}$ and verify that the atlas
consisting of the two charts $(\mathbb{S}^n\backslash{N},\sigma)$ and $(\mathbb{S}^n\backslash{S},
\tilde{\sigma})$ defines a smooth structure on $\mathbb{S}^n$
\end{enumerate}
\end{problem}
\begin{problem}
Let $M$ be a smooth n-manifold with boundary. Show that Int$M$ is a smooth n-manifold and $\partial M$
is a smooth $(n-1)$-manifold (bouth without boundary).
\end{problem}
\begin{problem}
Let $M=\overline{\mathbb{B}^n}$, the closed unit ball in $\R^n$. Show that $M$ is a manifold with
boundary and has a natural smooth structure s.t. its interior is the open unit ball with its standard
smooth structure.
\end{problem}